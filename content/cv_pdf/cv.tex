%%%%%%%%%%%%%%%%%%%%%%%%%%%%%%%%%%%%%%%%%
% "ModernCV" CV and Cover Letter
% LaTeX Template
% Version 1.11 (19/6/14)
%
% This template has been downloaded from:
% http://www.LaTeXTemplates.com
%
% Original author:
% Xavier Danaux (xdanaux@gmail.com)
%
% License:
% CC BY-NC-SA 3.0 (http://creativecommons.org/licenses/by-nc-sa/3.0/)
%
% Important note:
% This template requires the moderncv.cls and .sty files to be in the same 
% directory as this .tex file. These files provide the resume style and themes 
% used for structuring the document.
%
%%%%%%%%%%%%%%%%%%%%%%%%%%%%%%%%%%%%%%%%%

%----------------------------------------------------------------------------------------
%	PACKAGES AND OTHER DOCUMENT CONFIGURATIONS
%----------------------------------------------------------------------------------------

\documentclass[11pt,a4paper,serif,linkcolor=true]{moderncv} % Font sizes: 10, 11, or 12; paper sizes: a4paper, letterpaper, a5paper, legalpaper, executivepaper or landscape; font families: sans or roman

\moderncvstyle{casual}
\moderncvcolor{grey}
\usepackage[scale=0.8]{geometry} % Reduce document margins

\setlength{\hintscolumnwidth}{3cm} % Uncomment to change the width of the dates column
%\setlength{\makecvtitlenamewidth}{10cm} % For the 'classic' style, uncomment to adjust the width of the space allocated to your name

% own customizations
\usepackage[ngerman,english]{babel}
\usepackage[utf8]{inputenc}
\usepackage[default,osfigures,scale=0.95]{opensans}
\usepackage[T1]{fontenc}

\definecolor{linkblue}{RGB}{50,107,164} % Bootstrap style
\usepackage{url}
\urlstyle{same}
\AtBeginDocument{\hypersetup{colorlinks=true,urlcolor=linkblue,linkcolor=gray}}

%----------------------------------------------------------------------------------------
%	NAME AND CONTACT INFORMATION SECTION
%----------------------------------------------------------------------------------------

\firstname{Florian M.}
\familyname{Wagner}
\title{Curriculum vitae}
\email{mail@fwagner.info}
\homepage{www.fwagner.info}{www.fwagner.info}
\photo[150pt][0pt]{../static/fwagner.jpg}
\extrainfo{Last updated on \today}

%----------------------------------------------------------------------------------------

\begin{document}

\makecvtitle % Print the CV title

\section{Work Experience}
\cventry{since Nov. 2011}{Research associate}{GFZ German Research Centre for Geosciences, Section 6.3 - Geological Storage, Potsdam, Germany}{}{}{Research on geoelectrical CO$_2$ monitoring, field experiments, server administration}

\section{Education}
\cventry{1997--2006}{Abitur}{Friedrich-Wilhelm-Gymnasium}{Cologne}{}{General qualification for university entrance}  % arguments 3 to 6 can be left empty
\cventry{2006--2009}{Bachelor of Science}{RWTH Aachen University}{}{}{B.Sc. in Georesources Management}
\cventry{2009--2011}{Master of Science}{IDEA League: TU Delft, ETH Zurich, RWTH Aachen}{}{}{Joint M.Sc. in Applied Geophysics (\url{http://www.idealeague.org/geophysics})}
\cventry{since Nov. 2012}{PhD Student}{ETH Zurich}{}{}{}

\section{Practical Experience}
\cventry{March 2009}{Internship}{Trasswerke Meurin}{Andernach}{Germany}{Internship at Trasswerke Meurin Produktions- und Handelsgesellschaft mbH (\url{www.meurin.com}). Gained experience in the excavation of volcanic rocks for the production of quality construction materials including sieve analyses and strength tests.}
\cventry{June 2010}{Field Campaign}{ETH Zurich}{Kloten and Laegeren}{Switzerland}{Comprehensive geophysical field work incorporating data acquisition, processing and reporting. Applied measuring techniques included: Electrical Resistivity Tomography (ERT), Seismic Refraction Tomography (SRT), Ground Penetrating Radar (GPR), Electromagnetics (EM31 and EM38), Transient Electromagnetics (TEM) and Magnetics.}
\cventry{July 2010}{Internship}{DMT GmbH \& Co. KG}{Essen}{Germany}{Large-scale 3D seismic survey by DMT GmbH \& Co. KG (\url{www.dmt.de/en/home.html}) in Jointville, France.
\newline
\newline
Gained hands-on experiences in:
\begin{itemize}
\item Large-scale data acquisition
\item Reflection \& Refraction Seismics
\item Vertical Seismic Profiling (VSP)
\item Well Logging
\item Quality Control
\end{itemize}
}
\cventry{September 2013}{Research visit}{University of Alberta}{Edmonton}{Canada}{
I worked on acoustic and electrical analysis of reservoir sandstones in the Geomechanical Reservoir Experimental Facility (\url{www.geo-ref.ca}) of the University of Alberta in Edmonton, Canada, under supervision of Dr. Chalaturnyk and his team.
}

\section{Teaching experience}
\cvitem{Nov. 2012}{Introduction to Scientific Computing and Data Visualization using Python (Seminar talk at the German Research Centre for Geosciences)}
\cvitem{2013-2015}{Co-Supervision of the M.Sc. summer block course on Geophysical Field Work and Data Processing at ETH Zurich}
\cvitem{2012, 2015}{Supervision of two master theses within the IDEA League program}

\section{Scholarships / Awards}
\cvline{2009--2011}{Private scholarship from RWE Dea AG}
\cvline{2009--2011}{Co-financing from the education fund of North Rhine-Westphalia}
\cvline{Sep. 2012}{Best oral presentation award at the 2\textsuperscript{nd} Science Forum of the Helmholtz-Alberta-Initiative, Potsdam, Germany}
\cvline{Sep. 2013}{Best oral presentation award at the 3\textsuperscript{rd} Science Forum of the Helmholtz-Alberta-Initiative, Edmonton, Canada}
\cvline{Sep. 2014}{Best oral presentation award at the 4\textsuperscript{th} Science Forum of the Helmholtz-Alberta-Initiative, Edmonton, Canada}
\cvline{Apr. 2016}{1\textsuperscript{st} place in category "EGU Talk" at the 11\textsuperscript{th} annual GFZ PhD Day, Potsdam, Germany}

\section{Scientific interests}
\cvline{}{
\begin{itemize}
\item Geophysical monitoring of subsurface fluid migration
\item Tomographic experimental design
\item Numerical modeling and inversion
\item Data analysis and parallel computing
\item Scientific software development (\url{www.gimli.org})
\end{itemize}
}

\section{Languages}
\cvline{German}{Native}
\cvline{English}{Fluent (oral and written)}
\cvline{French}{Basic knowledge}

\section{Computer skills}
\cvcomputer{Programming}{Linux, Scientific Python, MATLAB}{Inversion}{BERT, GIMLi, PEST}
\cvcomputer{Numerical Modeling}{Experiences in TOUGH2, MODFLOW, SHEMAT}{Office / Writing}{LaTeX, Vim, MS Office}
\cvcomputer{Seismic Processing}{ProMAX}{Website development}{HTML, CSS, Photoshop, Illustrator}

\small
\input{publications}

\end{document}